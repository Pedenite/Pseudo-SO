\documentclass[conference]{IEEEtran}
%\IEEEoverridecommandlockouts
% The preceding line is only needed to identify funding in the first footnote. If that is unneeded, please comment it out.
\usepackage{cite}
\usepackage[brazil]{babel}
\usepackage[utf8]{inputenc}
\usepackage{amsmath,amssymb,amsfonts}
\usepackage{algorithmic}
\usepackage{graphicx}
\usepackage{textcomp}
\usepackage{xcolor}
\usepackage{lipsum}
\def\BibTeX{{\rm B\kern-.05em{\sc i\kern-.025em b}\kern-.08em
    T\kern-.1667em\lower.7ex\hbox{E}\kern-.125emX}}
\begin{document}

\title{IOS}

\author{\IEEEauthorblockN{Kesley Kenny Vasques Guimarães\IEEEauthorrefmark{1}, Pedro Henrique de Brito Agnes\IEEEauthorrefmark{2}}
\IEEEauthorblockA{\textit{Universidade de Brasília, Departmento de Ciência da Computação}\\
Brasília, Brasil \\
\{\IEEEauthorrefmark{1}180021231, \IEEEauthorrefmark{2}180026305\}@aluno.unb.br}
}

\maketitle

%%%%%%%%%%%%%%%%%%%%%%%%%%%%%%%%%%%%%%%%%%%%%%%%%%%%%%%%%%%%%%%%
%%%%%%%%%%%%%%%%%%%%%% START OF THE PAPER %%%%%%%%%%%%%%%%%%%%%%
%%%%%%%%%%%%%%%%%%%%%%%%%%%%%%%%%%%%%%%%%%%%%%%%%%%%%%%%%%%%%%%%%

\begin{abstract}
\lipsum[1]
\end{abstract}

\begin{IEEEkeywords}
IOS, Apple, Sistema operacional
\end{IEEEkeywords}

\section{Introdução}
Aqui fica a introdução. \lipsum[10] \lipsum[5]

\section{Sobre o IOS}
blablabla...\lipsum

\subsection{Exemplo de subsection...}

The IEEEtran class file is used to format your paper and style the text. All margins, 
column widths, line spaces, and text fonts are prescribed; please do not 
alter them. You may note peculiarities. For example, the head margin
measures proportionately more than is customary. This measurement 
and others are deliberate, using specifications that anticipate your paper 
as one part of the entire proceedings, and not as an independent document. 
Please do not revise any of the current designations.

\section{Conclusões}

\lipsum[1] \lipsum[2]

\begin{thebibliography}{00}
\bibitem{b1} foo, bar
\end{thebibliography}

\end{document}
